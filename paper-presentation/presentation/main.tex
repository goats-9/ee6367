% Set the document class and theme
\documentclass{beamer}
\usetheme{CambridgeUS}
\setbeamertemplate{caption}[numbered]

\usepackage{./presentation_macros}

% Add presentation data here 

% Text in the square brackets `[]' are shown in the footer. If not mentioned,
% then text in the curly braces `{}' are used as theme defaults.

\title[QUIC-FL]{QUIC-FL: Quick UnbIased Compression for Federated Learning\\
\small Ran Ben Basat, Shay Vargaftik, Amit Portnoy, Gil Einziger, Yaniv Ben-Itzhak, Michael Mitzenmacher}
\date{November 22, 2023}
\author{Gautam Singh}
\institute[]{Indian Institute of Technology Hyderabad}

% Presentation begins here

\begin{document}
    \maketitle
    \tableofcontents
    \section{Introduction}

    \begin{frame}
        \frametitle{The DME Problem}
        \begin{figure}[!ht]
            \begin{tikzpicture}[node distance = 5ex]
    \node[state] (C_1) {$C_1$}; 
    \node[state] (C_2) [right=of C_1] {$C_2$}; 
    \node (L) [right=of C_2] {$\ldots$};
    \node[state] (C_n) [right=of L] {$C_n$}; 
    \node[state,rectangle] (S) [above=of $(C_2.north)!0.5!(L.north)$] {$S$};
    \node (X_1) [below=of C_1] {$\vect{X_1} \in \bR^d$};
    \node (X_2) [below=of C_2] {$\vect{X_2}$};
    \node (X_n) [below=of C_n] {$\vect{X_n}$};
    \node (L1) [below=of L] {$\ldots$};
    \node (muhat) [above=of S] {$\hat{\vect{\mu}} \triangleq \frac{1}{n}\sum_{i=1}^n\hat{\vect{X_i}}$}; 
    \path[->]
    (C_1) edge  node [left=1ex] {$\vect{Y_1} \in \cbrak{0,1}^b$} (S)
    (C_2) edge  node [right=1ex] {$\vect{Y_2}$} (S)
    (C_n) edge  node [right=1ex] {$\vect{Y_n}$} (S)
    (S) edge node {} (muhat)
    (X_1) edge node {} (C_1)
    (X_2) edge node {} (C_2)
    (X_n) edge node {} (C_n);
\end{tikzpicture}
            \caption{Illustration of the DME Problem. Here, \(\hat{\vect{X_i}}\)
            denotes the server estimate for \(\vect{X_i}\).} 
            \label{fig:dme}
        \end{figure}
    \end{frame}

    \section{Preliminaries}
    \begin{frame}
        \frametitle{vNMSE and NMSE}
        \begin{definition}[vNMSE]
        The \emph{vector Normalized Mean Square Error} of \(\vect{x}\)
        is defined as
        \begin{equation}
            \vnmse\brak{\vect{x}} \triangleq \frac{\mean{\norm{\hat{\vec{x}}-\vec{x}}^2_2}}{\norm{\vect{x}}^2_2}.
            \label{eq:vnmse-def}
        \end{equation}
        \end{definition}
        \begin{definition}[NMSE]
        The \emph{Normalized Mean Square Error} in the case of the DME
        problem is defined as
        \begin{equation}
            \nmse \triangleq \frac{\mean{\norm{\hat{\vect{\mu}}-\vect{\mu}}^2_2}}{\frac{1}{n}\sum_{i=1}^n\norm{\vect{x_i}}^2_2} = \frac{\mean{\norm{\hat{\vect{\mu}}-\frac{1}{n}\sum_{i=1}^n\vect{x_i}}^2_2}}{\frac{1}{n}\sum_{i=1}^n\norm{\vect{x_i}}^2_2}.
            \label{eq:nmse-def}
        \end{equation}
        \end{definition}
    \end{frame}

    \begin{frame}
        \frametitle{Randomized Hadamard Transform}
        \begin{definition}[Walsh-Hadamard Matrix]
            The Walsh-Hadamard matrix \(\vect{H}_{2^k}\) with \(\vect{H}_1 = 
            \myvec{1}\) is recursively defined as
            \begin{equation}
                \vect{H}_{2^k} = \myvec{\vect{H}_{2^{k-1}} & \vect{H}_{2^{k-1}} \\ \vect{H}_{2^{k-1}} & \vect{H}_{2^{k-1}}}.
                \label{eq:Hl-def}
            \end{equation}
        \end{definition}
        \begin{definition}[Randomized Hadamard Transform]
            Let \(\vect{H}\in\cbrak{-1,+1}^{d\times d}\) be a Walsh-Hadamard 
            Matrix and \(\vect{D}\) be a diagonal matrix with uniform iid 
            Rademacher entries \emph{i.e.}, entries that are \(\pm 1\) with 
            equal probability. Then, the \emph{randomized Hadamard transform} 
            (RHT) of \(\vect{x}\in\bR^d\) is given by
            \begin{equation}
                \cR_{\vect{H}}\brak{\vect{x}} \triangleq \cR_{\vect{H}}\brak{\vect{x}} = \brak{\frac{1}{\sqrt{d}}\vect{HD}}\vect{x}.
                \label{eq:rht-def}
            \end{equation}
        \end{definition}
    \end{frame}

    \begin{frame}
        \frametitle{Properties of RHT}
        \begin{definition}{Inverse RHT}
            The \emph{inverse randomized Hadamard transform} of \(\vect{x}\) is given by
            \begin{equation}
                \cR_{\vect{H}}^{-1}\brak{x} \triangleq \cR_{\vect{H}}^{-1}\vect{x} = \cR_{\vect{H}}^\top\vect{x} = \brak{\frac{1}{\sqrt{d}}\vect{DH}}\vect{x}.
                \label{eq:inv-rht-def}
            \end{equation}
        \end{definition}
        \begin{lemma}
            Define \(F_{i,d}\brak{x}\) to be the CDF of 
            \(\frac{1}{\sigma}\cR_{\vect{H}}\brak{\vect{x}}_i\) and 
            \(\Phi\brak{x}\) the CDF of the standard normal distribution.
            Then, as \(d\to\infty\), \(\forall\ 1 \le j \le d\), 
            \begin{equation}
                F_{i,d}\brak{x_j} \xrightarrow{\mathrm{d}} \Phi\brak{x_j}.
                \label{eq:rht-norm-conv}
            \end{equation}
        \end{lemma}
    \end{frame}

    \begin{frame}
        \frametitle{Design Goals of QUIC-FL}
        \begin{enumerate}
            \item Less computational complexity compared to other methods, at
            \emph{both} client and server.
            \item Lesser amount of shared global as well as private randomness.
            \item Same (asymptotic) NMSE as other methods: \(\cO\brak{\frac{1}{n}}\).
        \end{enumerate}
    \end{frame}

    \section{Bounded Support Quantization (BSQ)}
    \begin{frame}
        \frametitle{BSQ Algorithm}
        \begin{enumerate}
            \item Uses a parameter \(p\in\rsbrak{\lbrak{0,1}}\), based on which 
            a threshold  \(\sbrak{-t_p,t_p}\) is chosen such that at most \(dp\)
            values lie outside this interval.
            \item \(\vect{x}\) separated into two parts: \emph{large}
            values that lie outside this interval and \emph{small} values
            that lie inside this interval.
            \item To encode \(\vect{x}\),
            \begin{enumerate}
                \item \emph{Large} values are sent exactly, along with their
                indices. Costs \(\log\binom{d}{dp} \approx 
                dp\log\brak{\frac{1}{p}}\) bits. Can use delta encoding or
                no encoding at cost of \(dp\ceil{\log d}\) bits (Authors assume
                \(p\log d << 1\)).
                \item \emph{Small} values are stochastically quantized and
                sent using \(b\) bits per value.
            \end{enumerate}
            \item To decode and compute \(\hat{\vect{x}}\), decode the \emph{large}
            values along with their indices. Then, estimate the quantized values.
        \end{enumerate}
    \end{frame}

    \begin{frame}
        \frametitle{Main Result of BSQ}
        \begin{theorem}
            BSQ, without further assumptions, admits a worst-case NMSE of
            \(\frac{1}{np\brak{2^b-1}^2}\) with \(\cO\brak{d}\) and
            \(\cO\brak{nd}\) complexity for encoding and decoding respectively.
        \end{theorem}
        \begin{alertblock}{Arbitrarily Large NMSE}
            Notice that the constant in the NMSE can be made arbitrarily large.
            For example, \(p = 2^{-10},\ b = 1\).
        \end{alertblock}
    \end{frame}
    
    \section{Distribution-Aware Unbiased Quantization}
    
    \section{The QUIC-FL Algorithm}
    
    \section{Results}
    
    \section{Further Improvements}
\end{document}
