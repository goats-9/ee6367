\documentclass[journal,12pt,twocolumn]{IEEEtran}
\usepackage{./report_macros}

\begin{document}
\vspace{3cm}
\title{QUIC-FL: A Report}
\author{Gautam Singh\\CS21BTECH11018}
\maketitle
\tableofcontents
\bigskip

\begin{abstract}
    This document is a report of the paper \cite{basat2023quicfl}. It 
    summarizes the main contributions of the authors and analyzes the results
    obtained. This report also lists out possible future research in the area
    of Federated Learning (FL).
\end{abstract}

\section{Problem Statement}
\label{sec:ps}
The authors of \cite{basat2023quicfl} consider the \textbf{Distributed Mean 
Estimation} problem, illustrated in \autoref{fig:dme}. Specifically, each 
client \(C_i\) sends data \(\vect{x}_i\), quantized as \(\vect{Y}_i\in
\cbrak{0,1}^b\), for \(1 \le i \le n\). Here, \(b\) represents the number
of bits used for quantization per client.

\begin{figure}[!ht]
    \begin{tikzpicture}[node distance = 5ex]
    \node[state] (C_1) {$C_1$}; 
    \node[state] (C_2) [right=of C_1] {$C_2$}; 
    \node (L) [right=of C_2] {$\ldots$};
    \node[state] (C_n) [right=of L] {$C_n$}; 
    \node[state,rectangle] (S) [above=of $(C_2.north)!0.5!(L.north)$] {$S$};
    \node (X_1) [below=of C_1] {$\vect{X_1} \in \bR^d$};
    \node (X_2) [below=of C_2] {$\vect{X_2}$};
    \node (X_n) [below=of C_n] {$\vect{X_n}$};
    \node (L1) [below=of L] {$\ldots$};
    \node (muhat) [above=of S] {$\hat{\vect{\mu}} \triangleq \frac{1}{n}\sum_{i=1}^n\hat{\vect{X_i}}$}; 
    \path[->]
    (C_1) edge  node [left=1ex] {$\vect{Y_1} \in \cbrak{0,1}^b$} (S)
    (C_2) edge  node [right=1ex] {$\vect{Y_2}$} (S)
    (C_n) edge  node [right=1ex] {$\vect{Y_n}$} (S)
    (S) edge node {} (muhat)
    (X_1) edge node {} (C_1)
    (X_2) edge node {} (C_2)
    (X_n) edge node {} (C_n);
\end{tikzpicture}
    \caption{An Illustration of the DME Problem.}
    \label{fig:dme}
\end{figure}

The server computes the estimates \(\hat{\vect{X}}_i\) using the obtained 
\(\vect{Y}_i\) as
\begin{equation}
    \hat{\vect{\mu}} \triangleq \frac{1}{n}\sum_{i=1}^n\hat{\vect{X}}_i.
    \label{eq:muhat-def}
\end{equation}

The aim of the DME problem is to estimate the true mean \(\vect{\mu} \triangleq
\frac{1}{n}\sum_{i=1}^n\vect{X}_i\) using \(\hat{\vect{\mu}}\) as defined in
\eqref{eq:muhat-def} with \emph{minimal error} (see \autoref{ssec:nmse}).

\section{Goals of the Paper}
\label{sec:goals}
The goals of \cite{basat2023quicfl} are to develop a quantization scheme that,
compared to other state-of-the-art methods (see \autoref{ssec:res-comp} for a 
list of such methods).

\begin{enumerate}
    \item Less computational complexity compared to other methods, at 
    \emph{both} client and server.
    \item Same (asymptotic) NMSE of \(\cO\brak{\frac{1}{n}}\).
    \item Same convergence rate.
    \item Better compression ratio.
\end{enumerate}

\section{Preliminaries}
\label{sec:prelims}
    \subsection{vNMSE and NMSE}
    \label{ssec:nmse}
    The main performance metric used to assess a quantization scheme is the
    \emph{squared error} of the estimated mean from the actual mean. To perform
    such an assessment, the authors define two quantities that will be useful.
    \begin{definition}[vNMSE]
        \label{def:vnmse}
        The \emph{vector Normalized Mean Square Error} of \(\vect{x}\)
        is defined as
        \begin{equation}
            vNMSE \triangleq \frac{\mean{\norm{\hat{\vec{x}}-\vec{x}}^2_2}}{\norm{\vect{x}}^2_2}.
            \label{eq:vnmse-def}
        \end{equation}
    \end{definition}
    \begin{definition}[NMSE]
        \label{def:nmse}
        The \emph{Normalized Mean Square Error} in the case of the DME
        problem is defined as
        \begin{equation}
            NMSE \triangleq \frac{\mean{\norm{\hat{\vect{\mu}}-\vect{\mu}}^2_2}}{\frac{1}{n}\sum_{i=1}^n\norm{\vect{x}_i}^2_2} = \frac{\mean{\norm{\hat{\vect{\mu}}-\frac{1}{n}\sum_{i=1}^n\vect{x}_i}^2_2}}{\frac{1}{n}\sum_{i=1}^n\norm{\vect{x}_i}^2_2}.
            \label{eq:nmse-def}
        \end{equation}
    \end{definition}
    From \autoref{def:vnmse} and \autoref{def:nmse}, we can perform simple 
    algebraic manipulations to obtain the following result.
    \begin{lemma}
    \end{lemma}
    \begin{proof}
    \end{proof}
    \subsection{Randomized Hadamard Transform}
    \label{ssec:rht}
\section{Bounded Support Quantization (BSQ)}
\label{sec:bsq}
\section{Distribution-Aware Unbiased Quantization}
\label{sec:dauq}
\section{The QUIC-FL Algorithm}
\label{sec:quicfl}
\section{Results}
\label{sec:res}
    \subsection{Complexity Analysis}
    \label{ssec:res-comp}
    \subsection{Accuracy}
    \label{ssec:res-acc}
\section{Future Works}
\label{sec:future}
\bibliography{references.bib}

\end{document}
