% To familiarize yourself with this template, the body contains
% some examples of its use.  Look them over.  Then you can
% run LaTeX on this file.  After you have LaTeXed this file then
% you can look over the result either by printing it out with
% dvips or using xdvi.
%

\documentclass[twoside]{article}
%\usepackage{soul}
\usepackage{./lecnotes_macros}


\begin{document}
%FILL IN THE RIGHT INFO.
%\lecture{**LECTURE-NUMBER**}{**DATE**}{**LECTURER**}{**SCRIBE**}
\lecture{18}{06 November 2023}{Shashank Vatedka}{Gautam Singh}
%\footnotetext{These notes are partially based on those of Nigel Mansell.}

%All figures are to be placed in a separate folder named ``images''

% **** YOUR NOTES GO HERE:

\section{DME With Known Distribution on Inputs}

Consider the DME problem where users \(U_i\) have samples \(X_i\) which are iid
from a distribution with unknown mean. Suppose that each user sends one bit of
information \(Y_i\). We require to minimize the maximum MSE, defined as

\begin{equation}
    \mathrm{MaxMSE} \triangleq \max_{\mu\in\bM}\abs{\hat{\mu}-\mu}^2.
    \label{eq:maxmse-def}
\end{equation}

\subsection{Location Family}

Let \(f_X\) denote a pdf with zero mean. Then, the location family 
corresponding to \(f_X\) is given by.

\begin{equation}
    \cL\brak{f_X} \triangleq \cbrak{f_{X,\mu}\brak{x} = f_X\brak{x-\mu}},\ \mu \in \bM.
    \label{eq:loc-fam-def}
\end{equation}

Thus, \(cL\brak{f_X}\) is a collection of pdfs where \(\mu\) is the \emph{only} 
unknown variable.

\subsection{DME From MSE}

In this situation, we have

\begin{align}
    \mse\brak{\mu} &= \frac{1}{m^2}\mean{\sum_{i=1}^m\brak{\hat{X}_i-\mu}}^2 \\
                   &= \frac{1}{m^2}\mean{\sum_{i=1}^m\brak{\hat{X}_i-\mu}^2 + \sum_{i=1}^m\sum_{\substack{j=1\\j\neq i}}^m\brak{\hat{X}_i-\mu}\brak{\hat{X}_j-\mu}} \\
                   &= \frac{1}{m}\mean{\brak{\hat{X}_1-\mu}^2} + \frac{m-1}{m}\brak{\mean{\hat{X}_1-\mu}}^2.
                   \label{eq:mse-mu}
\end{align}

Consider an encoder

\begin{equation}
    Y_i = 
    \begin{cases}
        1 & X_i \le \theta \\
        0 & X_i > \theta
    \end{cases}.
    \label{eq:onebit-enc}
\end{equation}

Thus, if \(F_X\) is invertible,

\begin{align}
    \frac{1}{m}\sum_{i=1}^mY_i &\xrightarrow{\mathrm{P}} F_X\brak{\theta-\mu} \\
    \implies \hat{\mu} &= \theta - F_X^{-1}\brak{\frac{1}{m}\sum_{i=1}^mY_i} 
    \label{eq:mu-est}
\end{align}

\end{document}

