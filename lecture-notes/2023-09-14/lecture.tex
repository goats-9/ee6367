% To familiarize yourself with this template, the body contains
% some examples of its use.  Look them over.  Then you can
% run LaTeX on this file.  After you have LaTeXed this file then
% you can look over the result either by printing it out with
% dvips or using xdvi.
%

\documentclass[twoside]{article}
%\usepackage{soul}
\usepackage{./lecnotes_macros}


\begin{document}
%FILL IN THE RIGHT INFO.
%\lecture{**LECTURE-NUMBER**}{**DATE**}{**LECTURER**}{**SCRIBE**}
\lecture{5}{14 September 2023}{Shashank Vatedka}{Gautam Singh}
%\footnotetext{These notes are partially based on those of Nigel Mansell.}

%All figures are to be placed in a separate folder named ``images''

% **** YOUR NOTES GO HERE:

\section{Deterministic Single-Bit Scalar Quantization}

Suppose we have to encode \(x\in\sbrak{0,1}\) using a single bit \(c\in\cbrak{0,1}\). Let the decoded output be \(\hat{x}\). We require to minimize the \emph{distortion} which is taken to be the maximum squared error

\begin{equation}
    \mathrm{Max-SE} = \max_{x\in\sbrak{0,1}}\brak{x-\hat{x}}^2.
    \label{eq:max-se}
\end{equation}

One such set of encoding and decoding functions is

\begin{align}
    \mathrm{Enc}\brak{x} &= 
    \begin{cases}
        0 & x \le \frac{1}{2} \\
        1 & x > \frac{1}{2}
    \end{cases}
    \label{eq:det-rd-enc} \\
    \mathrm{Dec}\brak{c} &= 
    \begin{cases}
        \frac{1}{4} & c = 0 \\
        \frac{3}{4} & c = 1
    \end{cases}.
    \label{eq:det-rd-dec}
\end{align}

This is called \emph{deterministic rounding} and has a maximum squared error of \(\frac{1}{16}\) as the absolute maximum error is \(\frac{1}{4}\).

\section{Random Single-Bit Scalar Quantization}

\begin{claim}
    The maximum MSE of a random quantizer under no shared randomness is at least \(\frac{1}{16}\).
\end{claim}

\begin{proof}
    In this case, \(c\) is a Bernoulli random variable and \(\hat{X}=\mathrm{Dec}\brak{c}\) is a random variable. Denote

    \begin{equation}
        \hat{X}_i \triangleq \mathrm{Dec}\brak{i},\ i \in \cbrak{0,1}.
        \label{eq:X-hat-i-def}
    \end{equation}

    WLOG, let \(\mean{\hat{X}_0} \le \mean{\hat{X}_1}\). Then,

    \begin{align}
        \mean{\hat{X}_0} &\le p_C\brak{0}\mean{\hat{X}_0} + p_C\brak{1}\mean{\hat{X}_1} \\
                         &= \mean{\hat{X}} \le \mean{\hat{X}_1}.
                         \label{eq:bw-x-hats}
    \end{align}

    Suppose that \(x=0\) is encoded. Then,
    
    \begin{align}
        \mean{\brak{x-\hat{X}}^2|x=0} &= \mean{\hat{X}^2|x=0} \\
                                      &\ge \brak{\mean{\hat{X}|x=0}}^2 \\
                                      &= \brak{\mean{\hat{X}_0|x=0}p_{c|x=0}\brak{0} + \mean{\hat{X}_1|x=0}p_{c|x=0}\brak{1}}^2 \\
                                      &\ge \brak{\mean{\hat{X}_0|x=0}}^2.
                                      \label{eq:rand-mse-lb-0}
    \end{align}

    If the claim is not true, then \eqref{eq:rand-mse-lb-0} gives

    \begin{equation}
        \mean{\hat{X}_0|x=0} \le \frac{1}{4}.
        \label{eq:xhat-0}
    \end{equation}

    Similarly, if \(x=1\) is encoded,

    \begin{align}
        \brak{\mean{1-\hat{X}|x=1}}^2 &= \brak{1-\mean{\hat{X}|x=1}}^2 \\
                                      &\ge \brak{1-\mean{\hat{X}_1|x=1}}^2.
                                      \label{eq:rand-mse-lb-1}
    \end{align}

    Again, if the claim does not hold,

    \begin{equation}
        \mean{\hat{X}_1|x=1} > \frac{3}{4}.
        \label{eq:xhat-1}
    \end{equation}

    Now, choosing \(x=\frac{1}{2}\), and using \eqref{eq:xhat-0} and \eqref{eq:xhat-1},
    \begin{align}
        \mean{\brak{\hat{X}-\frac{1}{2}}^2|x=\frac{1}{2}} &= \mean{\brak{\hat{X}_0 - \frac{1}{2}}^2|x=\frac{1}{2},\ c=0}p_c\brak(0) + \mean{\brak{\hat{X}_1-\frac{1}{2}}^2|x=\frac{1}{2},\ c=1}p_c\brak{1} \\
                                                          &\ge \brak{\mean{\hat{X}_0-\frac{1}{2}|x=\frac{1}{2},\ c=0}}^2p_c\brak{0} + \brak{\mean{\hat{X}_1-\frac{1}{2}|x=\frac{1}{2},\ c=1}}^2p_c\brak{1} \\
                                                          &\ge \brak{\frac{1}{4}-\frac{1}{2}}^2p_c\brak{0} + \brak{\frac{3}{4}-\frac{1}{2}}^2p_c\brak{1} \ge \frac{1}{16}
    \end{align}
\end{proof}

\end{document}
