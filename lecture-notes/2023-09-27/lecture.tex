% To familiarize yourself with this template, the body contains
% some examples of its use.  Look them over.  Then you can
% run LaTeX on this file.  After you have LaTeXed this file then
% you can look over the result either by printing it out with
% dvips or using xdvi.
%

\documentclass[twoside]{article}
%\usepackage{soul}
\usepackage{./lecnotes_macros}


\begin{document}
%FILL IN THE RIGHT INFO.
%\lecture{**LECTURE-NUMBER**}{**DATE**}{**LECTURER**}{**SCRIBE**}
\lecture{10}{27 September 2023}{Shashank Vatedka}{Gautam Singh}
%\footnotetext{These notes are partially based on those of Nigel Mansell.}

%All figures are to be placed in a separate folder named ``images''

% **** YOUR NOTES GO HERE:

\section{Unbiased Estimation in DRIVE}

\begin{claim}
    If \(\vec{R}\) is a uniform random rotation and \(s =
    \frac{\norm{\vec{x}}_2}{\norm{\vec{Rx}}_1}\), then DRIVE is unbiased.
\end{claim}

\begin{proof}
    We want to show that \(\mean{\hat{\vec{X}}} = \vec{x}\ \forall \vec{x} \in 
    \bR^d\). Consider
    \begin{equation}
        \vec{x}^\prime = \myvec{r & 0 & \ldots & 0}^\top
        \label{eq:xpr-def}
    \end{equation}
    where \(r = \norm{\vec{x}}_2\). Note that any vector \(\vec{x}\) can be 
    rotated to \(\vec{x}^\prime\), call the rotation matrix 
    \(\vec{R}_{\vec{x}\rightarrow\vec{x}^\prime}\). We see immediately that
    \begin{equation}
        \vec{R}_{\vec{x}\rightarrow\vec{x}^\prime} = \vec{R}_{\vec{x}^\prime\rightarrow\vec{x}}^{-1}.
        \label{eq:rrinv}
    \end{equation}
    Now,
    \begin{align}
        \hat{\vec{X}} &= s\vec{R}^\top\mathrm{sign}\brak{\vec{Rx}} \\
                      &= s\vec{R}_{\vec{x}\rightarrow\vec{x}^\prime}^{-1}\vec{R}_{\vec{x}\rightarrow\vec{x}^\prime}\vec{R}^\top\mathrm{sign}\brak{\vec{R}\vec{R}_{\vec{x}\rightarrow\vec{x}^\prime}^{-1}\vec{R}_{\vec{x}\rightarrow\vec{x}^\prime}\vec{x}} \\
                      &= s\vec{R}_{\vec{x}\rightarrow\vec{x}^\prime}^{-1}\vec{R}_{\vec{x}}^\top\mathrm{sign}\brak{\vec{R}_{\vec{x}}\vec{x}^\prime} \\
    \end{align}
    where we define
    \begin{equation}
        \vec{R}_{\vec{x}} \triangleq \vec{RR}_{\vec{x}\rightarrow\vec{x}^\prime}^{-1} = \myvec{\vec{r}_{\vec{x}}^{\brak{1}} & \ldots & \vec{r}_{\vec{x}}^{\brak{d}}}.
        \label{eq:Rx-def}
    \end{equation}
    By \eqref{eq:xpr-def}, we have
    \begin{equation}
        \vec{R}_{\vec{x}}\vec{x}^\prime = \norm{\vec{x}}_2\vec{r}_{\vec{x}}^{\brak{1}}.
        \label{eq:Rxpr-val}
    \end{equation}
    Define
    \begin{equation}
        \vec{y} \triangleq \mathrm{sign}\brak{\vec{R}_{\vec{x}}\vec{x}^\prime}.
        \label{eq:y-def}
    \end{equation}
    Thus,
    \begin{equation}
        \vec{R}_{\vec{x}}^\top\vec{y} = \myvec{\langle\vec{r}_{\vec{x}}^{\brak{1}}\rangle & \ldots & \langle\vec{r}_{\vec{x}}^{\brak{d}}\rangle}^\top.
        \label{eq:Rxy-val}
    \end{equation}
    But
    \begin{align}
        s &= \frac{\norm{\vec{x}}_2^2}{\norm{\vec{Rx}}_1} \\
          &= \frac{\norm{\vec{x}}_2^2}{\norm{\vec{R}_{\vec{x}}\vec{x}^\prime}_1} \\
          &= \frac{\norm{\vec{x}}_2^2}{\langle\vec{R}_{\vec{x}},\mathrm{sign}\brak{\vec{R}_{\vec{x}}\vec{x}^\prime}\rangle} \\
          &= \frac{\norm{vec{x}}_2^2}{\langle\vec{R}_{\vec{x}}\vec{x}^\prime,\vec{y}\rangle} \\
          &= \frac{\norm{vec{x}}_2}{\langle\vec{r}_{\vec{x}}^{\brak{1}},\vec{y}\rangle}
    \end{align}
    Thus,
    \begin{equation}
        \hat{\vec{X}} = \vec{R}_{\vec{x}\rightarrow\vec{x}^\prime}^{-1}\norm{\vec{x}}_2\myvec{\frac{\langle\vec{r}_{\vec{x}}^{\brak{1}}\rangle}{\langle\vec{r}_{\vec{x}}^{\brak{1}}\rangle} & \ldots & \frac{\langle\vec{r}_{\vec{x}}^{\brak{d}}\rangle}{\langle\vec{r}_{\vec{x}}^{\brak{1}}\rangle}}^\top.
    \end{equation}
    If \(\vec{y} = \vec{r}_{\vec{x}}^{\brak{1}}\), then we would have 
    \begin{equation}
        \hat{\vec{X}} = \vec{R}_{\vec{x}\rightarrow\vec{x}^\prime}^{-1}\norm{\vec{x}}_2\vec{e_1} = \vec{x}.
    \end{equation}
    Define
    \begin{equation}
        \bar{\vec{R}_{\vec{x}}} \triangleq \vec{R}\mathrm{diag}\brak{-1, 1, \ldots, 1}\vec{R}_{\vec{x}\rightarrow\vec{x}^\prime}^{-1}
        \label{eq:Rxbar-def}
    \end{equation}
    Thus, if we use \(\bar{\vec{R}_{\vec{x}}}\) instead, we would get
    \begin{equation}
        \hat{\vec{X}}^{\prime\prime} = \vec{R}_{\vec{x}\rightarrow\vec{x}^\prime}^{-1}\norm{\vec{x}}_2\myvec{1 & -\frac{\langle\vec{r}_{\vec{x}}^{\brak{2}}\rangle}{\langle\vec{r}_{\vec{x}}^{\brak{1}}\rangle} & \ldots -\frac{\langle\vec{r}_{\vec{x}}^{\brak{d}}\rangle}{\langle\vec{r}_{\vec{x}}^{\brak{1}}\rangle}}^\top.
        \label{eq:Xhat-dpr-def}
    \end{equation}
    We see that
    \begin{align}
        \mean{\hat{\vec{X}}^{\prime\prime}} &= \mean{\hat{\vec{X}}} \\
        \mean{\hat{\vec{X}}^{\prime\prime}+\hat{\vec{X}}} &= 2\vec{x}
    \end{align}
    which implies that
    \begin{equation}
        \mean{\hat{\vec{X}}} = \vec{x}
    \end{equation}
    as required.
\end{proof}

We generalize this scheme.

\begin{claim}
    Suppose we replace \(\mathrm{sign}\brak{\vec{Rx}}\) with any scalar quantizer
    \(Q\brak{\vec{Rx}}\), and take
    \begin{equation}
        s = \frac{\norm{\vec{x}}_2^2}{\langle\vec{Rx},Q\brak{\vec{Rx}}\rangle}.
        \label{eq:sq-def}
    \end{equation}
    Then, DRIVE is unbiased with MSE \( = \Theta\brak{\norm{\vec{x}}_2^2}\).
\end{claim}

However, if \emph{structured} random rotation is used, then we cannot make the
scheme unbiased by choosing \(s\).

Note that if \(\vec{R}\) is a uniform random rotation or a structured random
rotation, then \(\vec{y}_i = \vec{Rx}_i\) is approximately Gaussian distributed
for all \(i\) identically. That is,

\begin{equation}
    \abs{F_{Y_i}\brak{y}-F_G\brak{y}} \to 0\ d \to \infty\ \forall\ y \in \bR.
    \label{eq:y-gauss}
\end{equation}

\end{document}
