% To familiarize yourself with this template, the body contains
% some examples of its use.  Look them over.  Then you can
% run LaTeX on this file.  After you have LaTeXed this file then
% you can look over the result either by printing it out with
% dvips or using xdvi.
%

\documentclass[twoside]{article}
%\usepackage{soul}
\usepackage{./lecnotes_macros}


\begin{document}
%FILL IN THE RIGHT INFO.
%\lecture{**LECTURE-NUMBER**}{**DATE**}{**LECTURER**}{**SCRIBE**}
\lecture{14}{11 October 2023}{Shashank Vatedka}{Gautam Singh}
%\footnotetext{These notes are partially based on those of Nigel Mansell.}

%All figures are to be placed in a separate folder named ``images''

% **** YOUR NOTES GO HERE:

\textbf{Note:} \textit{
    The discussion started in Lecture 14 continues on in 
    Lecture 15. This document contains the content of both lectures clubbed
    together for the sake of continuity.
}

\section{Correlated Quantization}

We consider stochastic rounding in the case of \(n\) users. Let a random 
permutation \(\pi\) of \(\cbrak{0,1,\ldots,n-1}\) be shared among the users.
User \(i\) is given the number \(\pi_i\) (here the users are 0-indexed). Then,
each user generates \(\gamma_i \sim \mathrm{Unif}\sbrak{0,\frac{1}{m}}\).
Note that \(\gamma_i\) is \emph{private randomness}. Define

\begin{equation}
    U_i \triangleq \frac{\pi_i}{m} + \gamma_i.
    \label{eq:Ui-def}
\end{equation}

The CDF of \(U_i\), for \(u \in \sbrak{0,1}\), is

\begin{align}
    \pr{U_i\le u} &= \pr{\frac{\pi_i}{m}+ \gamma_i \le u} \\
                  &= \pr{\pi_i + m\gamma_i \le mu} \\
                  &= \pr{\pi_i < \floor{mu}} + \pr{\pi_i = \floor{mu},\ m\gamma_i \le \mu - \floor{mu}} \\
                  &= \frac{\floor{mu}}{m} + \frac{mu-\floor{mu}}{m} = u
                  \label{eq:ui-cdf}
\end{align}

and hence \(U_i \sim \mathrm{Unif}\sbrak{0,1}\).

Now, each user \(i\) transmits

\begin{equation}
    Y_i \triangleq \mathbbm{1}_{\cbrak{U_i<x_i}}.
    \label{eq:Yi-def}
\end{equation}

Clearly, this scheme is unbiased because \(\mean{Y_i} = x_i\). We now present
a claim for the MSE of this scheme.

\begin{claim}
    For the above scheme, the MSE is upper bounded by \(\frac{3}{m}\sigma_{md} + \frac{12}{m^2}\), where

    \begin{equation}
        \sigma_{md} \triangleq \frac{1}{m}\sum_{i=1}^m\abs{x_i-\bar{x}} \le \sqrt{\frac{1}{m}\sum_{i=1}^m\brak{x_i-\bar{x}}^2}.
        \label{eq:sigma-md-def}
    \end{equation}
\end{claim}

\begin{proof}
    Define

    \begin{align}
        z_i &\triangleq \frac{\floor{mx_i}}{m} \label{eq:zi-def} \\
        Y_i &\triangleq Q_i\brak{x_i} \label{eq:Yi-qi-def} \\
        Q_i\brak{t} &\triangleq \mathbbm{1}_{\cbrak{U_i<{t}}} \label{eq:Qi-def}
    \end{align}

    The MSE is given by
    \begin{align}
        \mathrm{MSE} &= \mean{\brak{\frac{1}{m}\sum_{i=1}^mY_i - \frac{1}{m}\sum_{i=1}^mx_i}^2} \\
                     &= \frac{1}{m^2}\mean{\brak{\sum_{i=1}^m\brak{Y_i-x_i}}^2} \\
                     &= \frac{1}{m^2}\mean{\brak{\sum_{i=1}^m\brak{\brak{x_i-z_i}+\brak{z_i-Q_i\brak{z_i}}+\brak{Q_i\brak{z_i}-Q_i\brak{x_i}}}}^2} \label{eq:sum-three} \\
                     &\le \frac{3}{m^2}\cbrak{\mean{\brak{\sum_{i=1}^m\brak{x_i-z_i}}^2}+\mean{\brak{\sum_{i=1}^m\brak{z_i-Q_i\brak{z_i}}}^2}+\mean{\brak{\sum_{i=1}^m\brak{Q_i\brak{z_i}-Q_i\brak{x_i}}}^2}}
    \end{align}

    where we use in \eqref{eq:sum-three} the inequality

    \begin{equation}
        \brak{\alpha+\beta+\gamma}^2 \le 3\brak{\alpha^2+\beta^2+\gamma^2}.
        \label{eq:sum-three-ineq}
    \end{equation}
    for reals \(\alpha,\beta,\gamma\).

    Note that
    \begin{equation}
        x_i-z_i = x_i - \frac{\floor{mx_i}}{m} = \frac{mx_i - \floor{mx_i}}{m} \le \frac{1}{m},
    \end{equation}
    hence
    \begin{equation}
        \mean{\brak{\sum_{i=1}^m\brak{x_i-z_i}}^2} \le 1.
        \label{eq:term-1}
    \end{equation}

    Now,

    \begin{align}
        \mean{\brak{\sum_{i=1}^mz_i-Q_i\brak{z_i}}^2} &= \sum_{i=1}^m\mean{\brak{z_i-Q_i\brak{z_i}}^2} + \sum_{i=1}^{m}\sum_{\substack{j=1\\j\neq i}}^{m}\mean{\brak{z_i-Q_i\brak{z_i}}\brak{z_j-Q_j\brak{z_j}}} \\
                                                      &= \sum_{i=1}^m\mean{\brak{z_i-Q_i\brak{z_i}}^2} + \sum_{i=1}^m\sum_{\substack{j=1\\j\neq m}}^m\brak{\mean{Q_i\brak{z_i}Q_j\brak{z_j}}-z_iz_j}
                                                      \label{eq:term-2-solve}
    \end{align}

    Note that

    \begin{align}
        Q_i\brak{z_i} &= \mathbbm{1}_{\cbrak{\pi_i+m\gamma_i<mz_i}} \\
                      &= \mathbbm{1}_{\cbrak{\pi_i<mz_i}}
                      \label{eq:Qi-alt-def}
    \end{align}

    and

    \begin{align}
        \mean{Q_i\brak{z_i}Q_j\brak{z_j}} &= \pr{\pi_i<mz_i,\ \pi_j<mz_j} \\
                                          &= \min\cbrak{z_i,z_j}\frac{m\max\cbrak{z_i,z_j}-1}{m-1} \\
                                          &= \frac{mz_iz_j-\min\cbrak{z_i,z_j}}{m-1} \\
                                          &= \frac{mz_iz_j}{m-1} - \frac{1}{m-1}\brak{\frac{z_i+z_j}{2}-\frac{\abs{z_i-z_j}}{2}}.
                                          \label{eq:exp-qiqj}
    \end{align}

    Using \eqref{eq:exp-qiqj} in \eqref{eq:term-2-solve},

    \begin{align}
        \mean{\brak{\sum_{i=1}^mz_i-Q_i\brak{z_i}}^2} &= \sum_{i=1}^mz_i\brak{1-z_i} + \sum_{i=1}\sum_{\substack{j=1\\j\neq i}}^m \brak{\frac{z_iz_j}{m-1} - \frac{z_i+z_j}{2\brak{m-1}} + \frac{\abs{z_i-z_j}}{2\brak{m-1}}} \\
                                                      &= \sum_{i=1}^mz_i\brak{1-z_i} + \frac{\brak{\sum_iz_i}^2-\sum_iz_i^2}{m-1} - \sum_iz_i + \sum_{i=1}^m\sum_{\substack{j=1\\j\neq i}}^m\frac{\abs{z_i-z_j}}{2\brak{m-1}} \\
                                                      &\le \frac{\brak{\sum_iz_i}^2}{m} - \sum_iz_i^2 + \sum_{i=1}^m\sum_{\substack{j=1\\j\neq i}}^m\frac{\abs{z_i-z_j}}{2\brak{m-1}} \\
                                                      &\le \sum_{\substack{j=1\\j\neq i}}^m\frac{\abs{z_i-z_j}}{2\brak{m-1}} \\
                                                      &\le \sum_{\substack{j=1\\j\neq i}}^m\frac{\abs{x_i-x_j}+\frac{1}{m}}{2\brak{m-1}} \\
                                                      &\le \sum_{i=1}^m\sum_{\substack{j=1\\j\neq i}}^m\frac{\abs{x_i-x_j}}{2\brak{m-1}} + \frac{1}{2}.
                                                      \label{eq:term-2}
    \end{align}

    Finally, note that

    \begin{equation}
        Q_i\brak{z_i} = 1 \implies U_i < z_i \le x_i \implies Q_i\brak{x_i} = 1
        \label{eq:ineq-qz-qx}
    \end{equation}

    and thus

    \begin{equation}
        \mean{\brak{\sum_{i=1}^m\brak{Q_i\brak{z_i}-Q_i\brak{x_i}}}^2} \le 2.
        \label{eq:term-3}
    \end{equation}

    Putting \eqref{eq:term-1}, \eqref{eq:term-2} and \eqref{eq:term-3} together,

    \begin{align}
        \mathrm{MSE} &\le \frac{3}{m^2}\brak{4 + \sum_{i=1}^m\sum_{\substack{j=1\\j\neq i}}^m\frac{\abs{x_i-x_j}}{2\brak{m-1}}} \\
                     &\le \frac{12}{m^2} + \frac{3}{m}\sigma_{md}
                     \label{eq:mse-claim-proof}
    \end{align}
    as desired. It is left as an exercise to show that

    \begin{equation}
        \sum_{i=1}^m\sum_{\substack{j=1\\j\neq i}}^m\frac{\abs{x_i-x_j}}{2m\brak{m-1}} \le \sigma_{md} = \frac{1}{m}\sum_{i=1}^m\abs{x_i-\bar{x}}.
        \label{eq:md-ineq}
    \end{equation}
\end{proof}

\end{document}
