% To familiarize yourself with this template, the body contains
% some examples of its use.  Look them over.  Then you can
% run LaTeX on this file.  After you have LaTeXed this file then
% you can look over the result either by printing it out with
% dvips or using xdvi.
%

\documentclass[twoside]{article}
%\usepackage{soul}
\usepackage{./lecnotes_macros}


\begin{document}
%FILL IN THE RIGHT INFO.
%\lecture{**LECTURE-NUMBER**}{**DATE**}{**LECTURER**}{**SCRIBE**}
\lecture{4}{13 September 2023}{Shashank Vatedka}{Gautam Singh}
%\footnotetext{These notes are partially based on those of Nigel Mansell.}

%All figures are to be placed in a separate folder named ``images''

% **** YOUR NOTES GO HERE:

\section{Lossy Compresssion, Quantization}

We require to communicate \(X^d\in\bR^d\). To do so, it is encoded as \(c^k\in\cbrak{0,1}^k\). Suppose the decoded quantity is \(\hat{X}^d\in\bR^d\). Here, we have \(k = dR\), that is, \(k\) is linearly dependent on \(d\). The \emph{average distortion} is given by

\begin{equation}
    L = \sum_{i=1}^k\mean{l\brak{X_i,\hat{X_i}}}
    \label{eq:avg-dist}
\end{equation}

and the \emph{mean-squared-error} is given by

\begin{equation}
    \mse = \sum_{i=1}^n\mean{\brak{X_i-\hat{X_i}}^2}.
    \label{eq:mse-def}
\end{equation}

\subsection{Scalar Quantization}

Consider the case when \(d=1\). Let \(X\sim f_X\). A possible encoding function could be

\begin{equation}
    c = \mathrm{Enc}\brak{X} =
    \begin{cases}
        1 & X \ge \mean{X} \\
        0 & X < \mean{X}
    \end{cases}.
    \label{eq:enc-possible}
\end{equation}

The best decoding function is

\begin{equation}
    \mathrm{Dec}\brak{X} =
    \begin{cases}
        \mean{X|X\ge\mean{X}} & c = 1 \\
        \mean{X|X<\mean{X}} & c = 0
    \end{cases}.
    \label{eq:dec-possible}
\end{equation}

\begin{theorem}
    \label{thm:mse-min}
    For any \(\alpha\in\bR\) and random variable \(X\),
    \begin{equation}
        \mean{\brak{X-\alpha}^2} \ge \var\brak{X}
        \label{eq:var-mse}
    \end{equation}
    with equality iff \(\alpha = \mean{X}\).
\end{theorem}

\begin{proof}
    We have,
    \begin{align}
        \mean{\brak{X-\alpha}^2} &= \mean{\sbrak{\brak{X-\mean{X}}+\brak{\mean{X}-\alpha}}^2} \\
                                 &= \var\brak{X} + \brak{\mean{X}-\alpha}^2
                                 \ge \var\brak{X}
    \end{align}
    with equality iff \(\alpha = \mean{X}\).
\end{proof}

Using \autoref{thm:mse-min}, we see that \eqref{eq:dec-possible} is indeed the minimum mean-squared error (MMSE) estimator of \(X\). This gives us the following.

\begin{theorem}
    \label{thm:opt-dec-int}
    Given intervals \(I_i, 1 \le i \le k\), the optimal decoder is
    \begin{equation}
        \hat{X} = \hat{x}_i = \mean{X|X\in I_i}.
        \label{eq:dec-k-bits}
    \end{equation}
    The points \(\hat{x}_i\) are called \textbf{reconstruction points}.
\end{theorem}

\subsection{Optimal Encoder Given Reconstruction Points}

Suppose the reconstruction points \(\cbrak{\hat{x}_i}_{i=1}^{2^k}\) are given. We require to design an encoder that minimizes the MSE.

\begin{theorem}
    \label{thm:opt-enc-rp}
    Given the reconstruction points \(\cbrak{\hat{x}_i}\), the encoder
    \begin{equation}
        g^{*} \triangleq \argmin_i\brak{x-\hat{x}_i}^2
        \label{eq:g-min-def}
    \end{equation}
    is optimal in the MSE sense.
\end{theorem}

\begin{proof}
    Consider any encoder \(g\). Then, by the definition \eqref{eq:g-min-def},
    \begin{equation}
        \mse\brak{g} - \mse\brak{g^{*}} = \int_{-\infty}^{\infty}f_X\brak{x}\sbrak{\brak{x-g\brak{x}}^2 - \brak{x-g^{*}\brak{x}}^2} \ge 0
    \end{equation}
    which implies \(\mse\brak{g} \ge \mse\brak{g^{*}}\). Additionally, the intervals are
    \begin{equation}
        I_j \triangleq \brak{\frac{\hat{x}_{j-1}+\hat{x}_j}{2},\ \frac{\hat{x}_j+\hat{x}_{j+1}}{2}}
        \label{eq:I-j-def}
    \end{equation}
    where we define \(\hat{x}_{-1} = -\infty\) and \(\hat{x}_{2^k+1} = \infty\).
\end{proof}

\section{Lloyd-Max Algorithm}

Using \autoref{thm:opt-dec-int} and \autoref{thm:opt-enc-rp}, we obtain the Lloyd-Max Algorithm as follows.

\begin{enumerate}
    \item Choose \(I_i, 1 \le i \le 2^k\).
    \item Choose the best \(\hat{x}_i\) for the \(I_i\).
    \item Choose the best \(I_i\) for the \(\hat{x}_i\).
    \item Repeat until the decrease in MSE is small enough.
\end{enumerate}

Note that at every iteration, the mean squared error does decrease. However, it may not converge to the global MMSE and it may converge to a local minima.

\end{document}
