% To familiarize yourself with this template, the body contains
% some examples of its use.  Look them over.  Then you can
% run LaTeX on this file.  After you have LaTeXed this file then
% you can look over the result either by printing it out with
% dvips or using xdvi.
%

\documentclass[twoside]{article}
%\usepackage{soul}
\usepackage{./lecnotes_macros}


\begin{document}
%FILL IN THE RIGHT INFO.
%\lecture{**LECTURE-NUMBER**}{**DATE**}{**LECTURER**}{**SCRIBE**}
\lecture{20}{09 November 2023}{Shashank Vatedka}{Gautam Singh}
%\footnotetext{These notes are partially based on those of Nigel Mansell.}

%All figures are to be placed in a separate folder named ``images''

% **** YOUR NOTES GO HERE:

\section{Mean Squared Error for Location Families}
From the previous lecture, we know that

\begin{equation}
    \hat{p}_m \triangleq \frac{1}{m}\sum_{i=1}^mY_i \xrightarrow{\mathrm{as}} F_X\brak{\theta_0-\mu}
    \label{eq:pm-as-conv}
\end{equation}
and
\begin{equation}
    \mean{Y_i} = F_X\brak{\theta_0-\mu}.
    \label{eq:mean-Yi}
\end{equation}

Since \(Y_i\) is a bernoulli random variable,
\begin{equation}
    \var\brak{Y_i} = F_X\brak{\theta_0-\mu}\brak{1-F_X\brak{\theta_0-\mu}}.
    \label{eq:var-Yi}
\end{equation}

However, from the Central Limit Theorem,
\begin{equation}
    \sqrt{m}\brak{\hat{p}_m - F_X\brak{\theta_0-\mu}} \xrightarrow{\mathrm{d}} \cN\brak{0,\sigma^2}.
    \label{eq:pm-clt}
\end{equation}

Considering
\begin{equation}
    \phi\brak{z} \triangleq \theta_0 - F_X^{-1}\brak{z}
    \label{eq:phi-def}
\end{equation}
we use the delta method and \eqref{eq:pm-clt} to obtain

\begin{equation}
    \sqrt{m}\brak{\hat{\mu}-\mu} \xrightarrow{\mathrm{d}} \phi^\prime\brak{\theta}Z.
    \label{eq:phi-prime-solve}
\end{equation}

However,
\begin{align}
    \frac{d\phi}{dt} &= -\diff{t}{\brak{F_X^{-1}\brak{t}}} \\
                     &= -\frac{1}{F_X^\prime\brak{F_X^{-1}\brak{\theta}}} \\
                     &= -\frac{1}{f_X\brak{\theta_0-\mu}}.
                     \label{eq:phi-drv}
\end{align}

Therefore,
\begin{equation}
    \sqrt{m}\brak{\hat{\mu}-\mu} \xrightarrow{\mathrm{d}} -\frac{1}{f_X\brak{\theta_0-\mu}}\cN\brak{0,\var{Y_i}}
    \label{eq:muhat-dconv}
\end{equation}
and the mean squared error asymptotically is (as \(m\to\infty\)),
\begin{equation}
    \alpha^2 \triangleq m\mse \to \frac{F_X\brak{\theta_0-\mu}\brak{1-F_X\brak{\theta_0-\mu}}}{f_X^2\brak{\theta_0-\mu}}.
    \label{eq:muhat-mse}
\end{equation}

From \eqref{eq:muhat-mse}, observe that if \(\theta_0\) is close to \(\mu\), 
then the MSE is small (assuming \(f_X\) has single peak at \(x = 0\) and is 
symmetric). 

\section{Controlling the MSE}

In \eqref{eq:muhat-mse}, the numerator can be arbitrarily large. Thus, we
introduce the following protocol to recitfy that.

\begin{enumerate}
    \item Select first \(m^{0.9}\) users, and send \(\mathbbm{1}_{\cbrak{X_i\le\theta_0}} = Y_i\).
    \item Set
    \begin{equation}
        \theta_n := \theta_0 - F_X^{-1}\brak{\frac{1}{m^{0.9}}}
    \end{equation}
    \item Remaining users send \(Y_i = \mathbbm{1}_{\cbrak{X_i\le\theta_n}}\).
    Server updates
    \begin{equation}
        \hat{\mu} = \theta_n - F_X^{-1}\brak{\frac{1}{m_1}\sum_{i=1}^{m_1}Y_i},
        \label{eq:srv-upd-dme}
    \end{equation}
    where \(m_1 = m - m^{0.9}\) represents the unselected users.
\end{enumerate}

Using the previous approach, for symmetric distributions and conditioned on
\(\theta_n \xrightarrow{\mathrm{as}} \mu\),
\begin{equation}
    \alpha^2 \to \frac{1}{4f_X^2\brak{0}}
\end{equation}

We can obtain the lower bound for the case where \(f_X\) is symmetric and
\(\log{f_X}\) is concave, then for any single-bit estimator, we have

\begin{equation}
    \mse \ge \frac{1}{4mf_X^2\brak{0} + I_0}.
    \label{eq:mse-lb-concave}
\end{equation}

\end{document}

