% To familiarize yourself with this template, the body contains
% some examples of its use.  Look them over.  Then you can
% run LaTeX on this file.  After you have LaTeXed this file then
% you can look over the result either by printing it out with
% dvips or using xdvi.
%

\documentclass[twoside]{article}
%\usepackage{soul}
\usepackage{./lecnotes_macros}


\begin{document}
%FILL IN THE RIGHT INFO.
%\lecture{**LECTURE-NUMBER**}{**DATE**}{**LECTURER**}{**SCRIBE**}
\lecture{6}{18 September 2023}{Shashank Vatedka}{Gautam Singh}
%\footnotetext{These notes are partially based on those of Nigel Mansell.}

%All figures are to be placed in a separate folder named ``images''

% **** YOUR NOTES GO HERE:

\section{Unbiased Quantization of Real Numbers}

The deterministic encoding scheme discussed before is \emph{biased}. We consider a \emph{randomized rounding} scheme where

\begin{align}
    \mathrm{Enc}\brak{x} &= \mathrm{Ber}\brak{x} \\
    \mathrm{Dec}\brak{c} &= c
    \label{eq:ber-rand}
\end{align}

Here, the cost is

\begin{align}
    \mathrm{Cost} &= \max_{x\in\sbrak{0,1}}\mean{\brak{x-c}^2} \\
                  &= \max_{x\in\sbrak{0,1}}x\brak{1-x} = \frac{1}{4}.
\end{align}

We claim that for any unbiased algorithm with no shared randomness, the cost is lower bounded by \(\frac{1}{4}\).

\section{Estimation Schemes With Shared Randomness}

Consider an encoder and decoder which share a uniform random variable \(U\in\sbrak{0,1}\). The encoder and decoder are defined as follows.

\begin{align}
    \mathrm{Enc}\brak{x} &= c =
    \begin{cases}
        1 & U \le x \\
        0 & U > x
    \end{cases} \\
    \mathrm{Dec}\brak{c} &= \hat{X} = c + U - \frac{1}{2}.
\end{align}

Clearly, \(\mean{\hat{X}} = x\), so the scheme is unbiased. The cost is

\begin{align}
    \mean{\brak{\hat{X}-x}^2} &= \mean{\cbrak{\brak{c-x}+\brak{U-\frac{1}{2}}}^2} \\
                              &= \var\brak{c} + \var\brak{U} + 2\mean{\brak{c-x}\brak{U-\frac{1}{2}}} \\
                              &= x\brak{1-x} + \frac{1}{12} + 2\sbrak{\mean{cU}-x\mean{U-\frac{1}{2}}-\frac{1}{2}\mean{c}} \\
                              &= x\brak{1-x} + \frac{1}{12} - x + 2\int_0^1cuf_U\brak{u}du \\
                              &= x\brak{1-x} + \frac{1}{12} - x + 2\int_0^xudu = \frac{1}{12}
\end{align}

and this scheme beats randomized rounding with a cost equal to the variance of \(U\sim\mathrm{Unif}\sbrak{0,1}\).

In summary,

\begin{table}[!ht]
    \centering
    \begin{tabular}{|c|c|c|}
        \hline
        \textbf{Shared Randomness} & \textbf{Biased} & \textbf{Unbiased} \\
        \hline
        No & \(\frac{1}{16}\) & \(\frac{1}{4}\) \\
        \hline
        Yes & \(0.0459\ldots\) & \(\frac{1}{12}\) \\
        \hline
    \end{tabular}
    \caption{Cost of using various estimation schemes.}
    \label{tab:cost-est}
\end{table}

\section{Generalization To More Than One Bit}

If we have \(k\) bits for quantization, then

\begin{enumerate}
    \item For \emph{deterministic rounding}, split \(\sbrak{0,1}\) into \(2^k\) equal sized intervals. The cost is \(\brak{\frac{1}{2^{k+1}}}^2\).
    \item For \emph{randomized rounding}, split into \(2^k-1\) equal intervals, so that we have \(2^k\) reconstruction points. If \(x\in\sbrak{l_i,r_i}\), then
        \begin{align}
            \mathrm{Enc}\brak{x} &= c^k =
            \begin{cases}
                l_i & \textrm{wp} \frac{x-l_i}{r_i-l_i} \\
                r_i & \textrm{else}
            \end{cases} \\
            \mathrm{Dec}\brak{c^k} &= \mathrm{Real}\brak{c^k}
        \end{align}
\end{enumerate}

\end{document}
