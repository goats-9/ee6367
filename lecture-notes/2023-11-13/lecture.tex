% To familiarize yourself with this template, the body contains
% some examples of its use.  Look them over.  Then you can
% run LaTeX on this file.  After you have LaTeXed this file then
% you can look over the result either by printing it out with
% dvips or using xdvi.
%

\documentclass[twoside]{article}
%\usepackage{soul}
\usepackage{./lecnotes_macros}


\begin{document}
%FILL IN THE RIGHT INFO.
%\lecture{**LECTURE-NUMBER**}{**DATE**}{**LECTURER**}{**SCRIBE**}
\lecture{21}{13 November 2023}{Shashank Vatedka}{Gautam Singh}
%\footnotetext{These notes are partially based on those of Nigel Mansell.}

%All figures are to be placed in a separate folder named ``images''

% **** YOUR NOTES GO HERE:

\section{General Single-Bit Quantization Schemes}

Any one-bit quantization scheme takes the form

\begin{equation}
    Y_i = \mathbbm{1}_{\cbrak{X_i \in \cA_i}}.
    \label{eq:one-bit-gen}
\end{equation}

For a fully distributed scheme, the \(\cA_i\) are fixed and independent of the
samples \(X_i\). However, in an interactive scheme the \(\cA_i\) can depend on
the \(X_i\).

\subsection{A Potential One-Bit Scheme}
The mean of a random variable \(X\) can be written in another form as

\begin{align}
    \mean{X} &= \int_{-\infty}^{\infty}xf_X\brak{x}dx \\
             &= \int_{0}^{\infty}\pr{X>t}dt - \int_{-\infty}^0\pr{X<t}dt \\
             &= \int_{0}^{\infty}\mean{\mathbbm{1}_{\cbrak{X>t}}}dt - \int_{-\infty}^0\mean{\mathbbm{1}_{\cbrak{X<t}}}dt \\
             &\approx \sum_{j=0}^{\alpha}\mean{\mathbbm{1}_{\cbrak{X > j\Delta}}}\Delta - \sum_{j=-\beta}^{0}\mean{\mathbbm{1}_{\cbrak{X < j\Delta}}}\Delta
             \label{eq:X-mean-alt}
\end{align}
Thus, we can create a potential scheme in which the \(j\)\textsuperscript{th} user
transmits \(\mathbbm{1}_{\cbrak{X_j < j\Delta}}\).

\section{Threshold Based Schemes}

In threshold-based schemes, we have \(\cA_i = (-\infty,\theta_i]\), and
\(Y_i = \mathbbm{1}_{\brak{X_i\le\theta_i}}\). We state the following claim.

\begin{theorem}
    For any threshold-based scheme and symmetric log-concave \(f_X\), we have
    \begin{equation}
        \sqrt{m}\brak{\hat{\mu}-\mu} \xrightarrow{\mathrm{d}} \frac{1}{\kappa\brak{\mu}}
    \end{equation}
    as \(m\to\infty\). Further,
    \begin{align}
        \lim_{m\to\infty}\mse &\rightarrow \frac{1}{\kappa\brak{\mu}} \\
        \kappa\brak{\mu} &\triangleq \int_{-\infty}^{\infty}\frac{f_X^2\brak{t-\mu}}{F_X\brak{t-\mu}F_X\brak{\mu-t}}\lambda\brak{t}dt \\
        \Lambda_m\brak{\tau} &\triangleq \frac{1}{m}\sum_{i=1}^m\mathbbm{1}_{\theta_i \le \tau} \\
        \lambda\brak{\tau} &\triangleq \diff{\tau}{\lim_{m\to\infty}\Lambda_m\brak{\tau}} = \frac{d\Lambda\brak{\tau}}{d\tau}.
        \label{eq:thm-defs}
    \end{align}
\end{theorem}

Note that \(\Lambda_m\brak{z}\) is the approximation of a CDF. Further,
if \(\theta_i\)  is uniform over \(\sbrak{-\beta,\alpha}\), then
\begin{equation}
    \kappa\brak{\mu} = \int_{-\beta}^{\alpha}\frac{f_X^2\brak{t-\mu}}{F_X\brak{t-\mu}F_X\brak{\mu-t}}\frac{1}{\alpha+\beta}dt.
    \label{eq:kappa-mu-unif}
\end{equation}

\end{document}

